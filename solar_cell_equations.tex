\documentclass[a4paper,11pt]{article}

\usepackage[margin=2.5cm]{geometry}
\usepackage{mathpazo}
\usepackage{hyperref}
\usepackage[doi=false,style=authoryear-comp,dashed=false,firstinits=true]{biblatex}
\bibliography{bibliography}
\usepackage{amsmath}

\renewcommand{\vec}[1]{\mathbf{#1}}

\begin{document}

\section{Spatial discretisation}

Here we attempt to derive the weak form of the solar cell model given in the
supplement to \cite{Calado2016}. We refer to equations from that supplement
with the prefix s. In correspondence with the authors, it emerges that
(s17), (s18), and (s20-22) have sign errors. These are (hopefully) corrected here.
We generalise the equations to arbitrary dimension. 

Let $\Omega = \Omega_n \cup \Omega_i \cup \Omega_p$ be the domain, and let
$\Gamma$ be the domain boundary, with the following boundary conditions
defined:
\begin{gather}
  J_n\cdot\vec{n} = 0 \quad\textrm{on }\Gamma\backslash\Gamma_n\\
  n = n_0 \quad\textrm{on }\Gamma_n\\
  J_p\cdot\vec{n} = 0 \quad\textrm{on }\Gamma\backslash\Gamma_p\\
  p = p_0 \quad\textrm{on }\Gamma_p\\
  J_a\cdot\vec{n} = 0 \quad\textrm{on }\Gamma\backslash\Gamma_a\\
  a = a_0 \quad\textrm{on }\Gamma_a\\
  \nabla V\cdot\vec{n} = 0 \quad\textrm{on }\Gamma\backslash\Gamma_V\\
  V = V_0 \quad\textrm{on }\Gamma_V
\end{gather}
where $\vec{n}$ is the outward pointing normal to the domain.

Let's start with the equation for negative charge carrier density $n$. From (s16) we have:
\begin{align}
  \frac{\partial n}{\partial t} = \frac{1}{q}\nabla\cdot J_n + G_n - U_n \quad \textrm{on
  }\Omega
\end{align}
where from s(11):
\begin{align}
  J_n =  q\mu_e(-n\nabla V + k_b T \nabla n) \label{nflux}
\end{align}

We choose a suitable function space $\mathcal{N}\equiv \mathcal{N}_0 \oplus \mathcal{N}_{\Gamma}$, multiply by a test function $\hat{n}$, and
integrate by parts. The problem becomes becomes find $n\in \mathcal{N}$ such that:
\begin{align}
  \int_\Omega \frac{\partial n}{\partial t}\hat{n} \,\mathrm{d}x = 
    \int_\Omega -\frac{J_n}{q} \cdot \nabla\hat{n} + (G_n - U_n)\hat{n} \,\mathrm{d}x
   + \int_\Gamma \frac{J_n}{q}  \hat{n} \cdot \vec{n} \,\mathrm{d}s \quad
  \forall \hat{n} \in \mathcal{N}_0.
\end{align}
However the surface term vanishes due to the boundary
conditions. Substituting in \eqref{nflux}, we have:
\begin{align}
  \int_\Omega \frac{\partial n}{\partial t}\hat{n} \,\mathrm{d}x = 
  \int_\Omega \mu_e\left(n \nabla V  
  - \frac{k_b}{q} T  \nabla n \right)\cdot \nabla \hat{n} 
  + \left(G_n - U_n\right)\hat{n} \,\mathrm{d}x \quad
  \forall \hat{n} \in \mathcal{N}_0.\label{eq:n}
\end{align}
Similarly, but accounting for the opposite drift direction of positive
charges, we have find
$p\in \mathcal{P}\equiv\mathcal{P}_0 \oplus\mathcal{P}_{\Gamma_p}$ such
that:
\begin{align}
  \int_\Omega \frac{\partial p}{\partial t} \hat{p}\,\mathrm{d}x = 
  \int_\Omega \mu_h  \left(-p \nabla V  
  - \frac{k_b}{q} T  \nabla p\right) \cdot \nabla \hat{p} 
  + \left(G_p - U_p\right)\hat{p} \,\mathrm{d}x \quad
  \forall \hat{p} \in \mathcal{P}_0;\label{eq:p}
\end{align}
and find $a\in \mathcal{A}\equiv\mathcal{A}_0 \oplus\mathcal{A}_{\Gamma_a}$ such that:
\begin{align}
  \int_\Omega \frac{\partial a}{\partial t} \hat{a}\,\mathrm{d}x = 
  \int_{\Omega_i} \mu_a  \left(-a \nabla V 
  - \frac{k_b}{q} T  \nabla a\right) \cdot \nabla \hat{a} 
  \,\mathrm{d}x \quad
  \forall \hat{a} \in \mathcal{A}_0. \label{eq:a}
\end{align}
Note the restriction to the intrinsic region in the RHS of
\eqref{eq:a}. Finally, if we define $\mathcal{V} \equiv \mathcal{V}_0 +
\mathcal{V}_{\Gamma_V}$ and integrate (s20-s22) by parts over the doman, we
have the problem find $v\in \mathcal{V}$ such that:
\begin{align}
  \int_\omega \nabla V \cdot \nabla\hat{V} \,\mathrm{d}x =
  \frac{q}{\epsilon_0\epsilon_r}\left(\int_{\Omega} n - p \,\mathrm{d}x +
  \int_{\Omega_p} N_A \hat{V} \,\mathrm{d}x + 
  \int_{\Omega_i} (-a + N_{\mathrm{ion}}) \hat{V} \,\mathrm{d}x + 
  \int_{\Omega_n} - N_D \hat{V} \,\mathrm{d}x \right) \quad\forall \hat{V}\in\mathcal{V}_0.
\end{align}
Note that the forcing and coupling terms apply over restricted parts
of the domain.
 
\subsection{Finite element spaces}

Since $n$, $p$, $a$ and $V$ are all intrinsic quantities, it appears
reasonable to represent them all in H1. That is to say we can try using
straightforward continuous finite element spaces for all of them. In the
first instance I'd try P1 (piecewise linear) elements. We can try getting
flashy later.

A potential issue with these elements is that the advection term (i.e. the
drift term) is subject to overshoots with these discretisations. Whether
this proves to be an issue in practice will depend on the relative sizes of
the drift and diffusion terms. If this does prove to be an issue, we'll
probably want to employ Petrov-Galerkin for the drift term.

\section{Time}

$p$ and $n$ move vastly more quickly than does $a$ (about $10^{13}$ faster
according to the values of $\mu_a$, $\mu_e$, and $\mu_h$). Accordingly, the
system as written will be impossibly stiff. A reasonable approximation is to
assume that $p$ and $n$ move infinitely quickly. This amounts to dropping
the time derivatives in \eqref{eq:n} and \eqref{eq:p}.

$a$ can be timestepped using any appropriate ODE solver. The most obvious
candidates would be the implicit midpoint rule or a Runge-Kutta scheme.

\printbibliography

\end{document}
